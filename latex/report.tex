\documentclass[12pt,a4paper]{report}  %紙張設定

%=------------------文件資料----------------------=%
\title{雲端開發與運維報告書}
\author{Author: lkh}
\date{Date: 2024.05.26}

%=------------------模組----------------------=%
\usepackage{xeCJK}%中文字體模組
%\setCJKmainfont{標楷體} %設定中文字體 本地使用
\setCJKmainfont{MoeStandardKai.ttf} %遠端使用
%\newfontfamily\sectionef{Times New Roman}%設定英文字體 本地使用
\newfontfamily\sectionef{Nimbus Roman} %遠端使用
\usepackage{enumerate}
\usepackage{amsmath,amssymb}%數學公式、符號
\usepackage{amsfonts} %數學簍空的英文字
\usepackage{graphicx, subfigure}%圖形
\usepackage{fontawesome5} %引用icon
\usepackage{type1cm} %調整字體絕對大小
\usepackage{textpos} %設定文字絕對位置
\usepackage[top=2.5truecm,bottom=2.5truecm,
left=3truecm,right=2.5truecm]{geometry}
\usepackage{titlesec} %目錄標題設定模組
\usepackage{titletoc} %目錄內容設定模組
\usepackage{textcomp} %表格設定模組
\usepackage{multirow} %合併行
%\usepackage{multicol} %合併欄
\usepackage{CJK} %中文模組
\usepackage{CJKnumb} %中文數字模組
\usepackage{wallpaper} %浮水印
\usepackage{listings} %引用程式碼
\usepackage{hyperref} %引用url連結
\usepackage{setspace}
\usepackage{lscape}%設定橫式
\lstset{language=Python, %設定語言
		basicstyle=\fontsize{10pt}{2pt}\selectfont, %設定程式內文字體大小
		frame=lines,	%設定程式框架為線
}
%\usepackage{subcaption}%副圖標
\graphicspath{{./../images/}} %圖片預設讀取路徑
\usepackage{indentfirst} %設定開頭縮排模組
\renewcommand{\figurename}{\Large 圖.} %更改圖片標題名稱
\renewcommand{\tablename}{\Large 表.}
\renewcommand{\lstlistingname}{\Large 程式.} %設定程式標示名稱
\hoffset=0mm %調整左右邊界
\voffset=0mm %調整上下邊界
\setlength{\parindent}{3em}%設定首行行距縮排
\usepackage{appendix} %附錄
\usepackage{diagbox}%引用表格
\usepackage{multirow}%表格置中
%\usepackage{number line}
\newcommand{\thirty}{\fontsize{30pt}\baselineskip\selectfont} %字體大小30pt
\newcommand{\twentyeight}{\fontsize{27pt}{\baselineskip}\selectfont} %字體大小28pt
\newcommand{\twentysix}{\fontsize{26pt}{\baselineskip}\selectfont} %字體大小26pt
\newcommand{\twentyfour}{\fontsize{24pt}{\baselineskip}\selectfont} %字體大小24pt
\newcommand{\twentytwo}{\fontsize{22pt}{\baselineskip}\selectfont} %字體大小22pt
\newcommand{\twenty}{\fontsize{20pt}{\baselineskip}\selectfont }%字體大小20pt
\newcommand{\eighteen}{\fontsize{18pt}{\baselineskip}\selectfont} %字體大小18pt
\newcommand{\sixteen}{\fontsize{16pt}{\baselineskip}\selectfont} %字體大小16pt
\newcommand{\fourteen}{\fontsize{14pt}{\baselineskip}\selectfont} %字體大小14pt
\newcommand{\twelve}{\fontsize{12pt}{\baselineskip}\selectfont} %字體大小12pt
%=------------------更改標題內容----------------------=%
\titleformat{\chapter}[hang]{\center\sectionef\fontsize{20pt}{1pt}\bfseries}{\LARGE 第\CJKnumber{\thechapter}章}{1em}{}[]
\titleformat{\section}[hang]{\sectionef\fontsize{18pt}{2.5pt}\bfseries}{{\thesection}}{0.5em}{}[]
\titleformat{\subsection}[hang]{\sectionef\fontsize{18pt}{2.5pt}\bfseries}{{\thesubsection}}{1em}{}[]
%=------------------更改目錄內容-----------------------=%
\titlecontents{chapter}[11mm]{}{\sectionef\fontsize{18pt}{2.5pt}\bfseries\makebox[3.5em][l]
{第\CJKnumber{\thecontentslabel}章}}{}{\titlerule*[0.7pc]{.}\contentspage}
\titlecontents{section}[18mm]{}{\sectionef\LARGE\makebox[1.5em][l]
{\thecontentslabel}}{}{\titlerule*[0.7pc]{.}\contentspage}
\titlecontents{subsection}[4em]{}{\sectionef\Large\makebox[2.5em][l]{{\thecontentslabel}}}{}{\titlerule*[0.7pc]{.}\contentspage}
%=----------------------章節間距----------------------=%
\titlespacing*{\chapter} {0pt}{0pt}{18pt}
\titlespacing*{\section} {0pt}{12pt}{6pt}
\titlespacing*{\subsection} {0pt}{6pt}{6pt}
%=----------------------標題-------------------------=%             
\begin{document} %文件
\sectionef %設定英文字體啟用
\vspace{12em}
\begin{titlepage}%開頭
\begin{center}   %標題  
\makebox[1.5\width][s] %[s] 代表 Stretch the interword space in text across the entire width
{\fontsize{24pt}{2.5pt}國立雲林科技大學}\\[18pt]
\sectionef\fontsize{24pt}{1em}\selectfont\textbf
{
\vspace{0.5em}
雲端開發與運維報告書}\\[18pt]
%設定文字盒子 [方框寬度的1.5倍寬][對其方式為文字平均分分布於方框中]\\距離下方18pt
\vspace{6em} %下移
\fontsize{24pt}{1pt}\selectfont\textbf{無伺服器架構下的開發與運維流程}\\
\vspace{1em}
\sectionef\fontsize{20pt}{1em}\selectfont\textbf
{
\vspace{0.5em}
Serverless architecture under the DevOps process}
 \vspace{5em}
%=---------------------參與人員-----------------------=%             
\end{center}
\begin{flushleft}
\begin{LARGE}

\hspace{30mm}\makebox[5cm][s]
{學\quad 生:\quad 林\quad 冠\quad 澔\quad(A11223032)}
\\[6pt]
\hspace{30mm}\makebox[5cm][s]
{\hspace{36.5mm}林\quad 家\quad 明\quad(B10923005)}\\[6pt]
\hspace{30mm}\makebox[5cm][s]
{\hspace{36.5mm}夏\quad 侯\quad 育\quad(B10923020)}\\[6pt]
\hspace{30mm}\makebox[5cm][s]
{\hspace{36.5mm}劉\quad 家\quad 豐\quad(B10923046)}\\[6pt]
\hspace{30mm}\makebox[5cm][s]
{\hspace{36.5mm}王\qquad \quad 瓅\quad(B11100028)}\\[6pt]
%設定文字盒子[寬度為5cm][對其方式為文字平均分分布於方框中]空白距離{36.5mm}\空白1em
\end{LARGE}
\end{flushleft}
\vspace{8em}
\fontsize{18pt}{2pt}\selectfont\centerline{\makebox[\width][s]
{中華民國\hspace{1em} 
113 \quad 年\quad 6\quad 月}}
\end{titlepage}
\newpage


%=------------------------摘要-----------------------=%
%\input{abstract.tex}
\newpage
%=------------------------目錄----------------------=%
\begin{spacing}{1.5}
\pagenumbering{roman}
\setcounter{page}{1}  %設定頁數
\renewcommand{\contentsname}{\centerline{\fontsize{24pt}{\baselineskip}\selectfont\textbf{目\quad 錄}}}
\tableofcontents  %目錄產生
\end{spacing}
\newpage
%=------------------圖表目錄產生----------------------=%
%\renewcommand{\listfigurename}{\centerline{\fontsize{18pt}{\baselineskip}\selectfont\textbf{圖\quad 目\quad 錄 }}}
%\newcommand{\loflabel}{圖} %定義\loflabel 文字為圖
%\renewcommand{\numberline}[1]{\loflabel~#1\hspace*{0.5em}}
%\listoffigures
%\newcommand{\captioname}{圖}
%\newpage
%\renewcommand{\listtablename}{\centerline{\fontsize{18pt}{\baselineskip}\selectfont\textbf{表\quad 目\quad 錄 }}}
%\newcommand{\lotlabel}{表} %定義\lotlabel 文字為表
%\renewcommand{\numberline}[1]{\lotlabel~#1\hspace*{0.5em}}
%\listoftables

%\end{center}
%=-------------------------內容----------------------=%
\setcounter{chapter}{0}
\chapter{簡介}
\pagenumbering{arabic}%數字頁數
\setcounter{page}{1}  %設定頁數
\renewcommand{\baselinestretch}{10} %設定行距
\section{無伺服器架構的概念}
\par
\renewcommand{\baselinestretch}{1} %設定行距
\twelve XXX\\
\par
\renewcommand{\baselinestretch}{1} %設定行距
\twelve XXX\\
\par
\renewcommand{\baselinestretch}{1} %設定行距
\twelve XXX
\par

\renewcommand{\baselinestretch}{20} %設定行距
\section{開發與運維的概念}
\par
\renewcommand{\baselinestretch}{1} %設定行距
\twelve XXX\\
\par
\renewcommand{\baselinestretch}{1} %設定行距
\twelve XXX
\begin{enumerate}
	\item XXX
	\item XXX
\end{enumerate}
\par

\renewcommand{\baselinestretch}{20} %設定行距
\section{無伺服器架構下的開發與運維優勢}
\par
\renewcommand{\baselinestretch}{1} %設定行距
\twelve XXX
\par

\chapter{Serverless架構的優勢與挑戰}
\renewcommand{\baselinestretch}{10} %設定行距
\section{優勢}
\par
\renewcommand{\baselinestretch}{1} %設定行距
\twelve XXX\\
\par
\renewcommand{\baselinestretch}{1} %設定行距
\twelve XXX\\
\par
\renewcommand{\baselinestretch}{1} %設定行距
\twelve XXX
\par

\renewcommand{\baselinestretch}{20} %設定行距
\section{挑戰}
\par
\renewcommand{\baselinestretch}{1} %設定行距
\twelve XXX\\
\par
\renewcommand{\baselinestretch}{1} %設定行距
\twelve XXX
\begin{enumerate}
	\item XXX
	\item XXX
\end{enumerate}
\par

\renewcommand{\baselinestretch}{20} %設定行距
\section{XXX}
\par
\renewcommand{\baselinestretch}{1} %設定行距
\twelve XXX
\par

\chapter{開發與運維在無伺服器架構中的腳色}
\renewcommand{\baselinestretch}{10} %設定行距
\section{持續整合}
\par
\renewcommand{\baselinestretch}{1} %設定行距
\twelve XXX\\
\par
\renewcommand{\baselinestretch}{1} %設定行距
\twelve XXX\\
\par
\renewcommand{\baselinestretch}{1} %設定行距
\twelve XXX
\par

\renewcommand{\baselinestretch}{20} %設定行距
\section{持續交付}
\par
\renewcommand{\baselinestretch}{1} %設定行距
\twelve XXX\\
\par
\renewcommand{\baselinestretch}{1} %設定行距
\twelve XXX
\begin{enumerate}
	\item XXX
	\item XXX
\end{enumerate}
\par

\renewcommand{\baselinestretch}{20} %設定行距
\section{監控與調試}
\par
\renewcommand{\baselinestretch}{1} %設定行距
\twelve XXX
\par

\chapter{實際案例}
\renewcommand{\baselinestretch}{10} %設定行距
\section{案例一}
\par
\renewcommand{\baselinestretch}{1} %設定行距
\twelve XXX\\
\par
\renewcommand{\baselinestretch}{1} %設定行距
\twelve XXX\\
\par
\renewcommand{\baselinestretch}{1} %設定行距
\twelve XXX
\par

\renewcommand{\baselinestretch}{20} %設定行距
\section{案例二}
\par
\renewcommand{\baselinestretch}{1} %設定行距
\twelve XXX\\
\par
\renewcommand{\baselinestretch}{1} %設定行距
\twelve XXX
\begin{enumerate}
	\item XXX
	\item XXX
\end{enumerate}
\par

\renewcommand{\baselinestretch}{20} %設定行距
\section{XXX}
\par
\renewcommand{\baselinestretch}{1} %設定行距
\twelve XXX
\par

\chapter{Github Action自動生成報告}
\renewcommand{\baselinestretch}{10} %設定行距
\section{工作流程}
\par
\renewcommand{\baselinestretch}{1} %設定行距
\twelve XXX\\
\par
\renewcommand{\baselinestretch}{1} %設定行距
\twelve XXX\\
\par
\renewcommand{\baselinestretch}{1} %設定行距
\twelve XXX
\par

\renewcommand{\baselinestretch}{20} %設定行距
\section{版本控制和發布}
\par
\renewcommand{\baselinestretch}{1} %設定行距
\twelve XXX\\
\par
\renewcommand{\baselinestretch}{1} %設定行距
\twelve XXX
\begin{enumerate}
	\item XXX
	\item XXX
\end{enumerate}
\par

\renewcommand{\baselinestretch}{20} %設定行距
\section{部屬和共享}
\par
\renewcommand{\baselinestretch}{1} %設定行距
\twelve XXX
\par

\renewcommand{\baselinestretch}{20} %設定行距
\section{比較與Word優劣}
\par
\renewcommand{\baselinestretch}{1} %設定行距
\twelve XXX
\par

%=---------------------尾頁----------------------=%
\newpage
%=----------------書背----------------------=%
\pagestyle{empty}%設定沒有頁眉和頁腳
\begin{center}
\fontsize{0.001pt}{1pt}\selectfont
\vspace{-7em}
\fontsize{26pt}{20pt}\selectfont 【X】 \\
\fontsize{18pt}{16pt}\selectfont
\vspace{0.5em}
分\\
類\\
編\\
號\\
\vspace{0.5em}
\hspace{-0.5em}:\\
\vspace{0.5em}
\rotatebox[origin=cc]{270}{\sectionef\LARGE \textbf{2024-06-雲端開發與運維}}\\ %旋轉
\vspace{0.5em}
無\\伺\\服\\器\\架\\構\\下\\的\\開\\發\\與\\運\\維\\流\\程\\
\vspace{1em}
一\\
一\\
三\\
%=---------------附錄-----------------=%
%\addcontentsline{toc}{chapter}{附錄} %新增目錄名稱
%\input{10_appendix.tex}
%\newpage
%\end{document}